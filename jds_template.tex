% The `linenumber` option is necessary for ease of reviewer referencing
\documentclass[linenumber]{jdsart} 

\volume{0}
\issue{0}
\pubyear{2022}
\doi{0000}

%% Article types correspond to different sections of the JDS
%% Choose one that's most suitable for your submission
% \articletype{Statistical Data Science}
% \articletype{Data Science in Action}
% \articletype{Compiuting in Data Science}
% \articletype{Data Sccience Review}
% \articletype{Philosophy of Data Science}
% \articletype{Data Science Education}
\articletype{Data Science Conversation}



%startlocaldefs
%% Put your local definitions here 

% local macros
%\newtheorem{thm}{Theorem}
%\newtheorem{prop}{Proposition}
%\newtheorem{cor}{Corollary}
%\newtheorem{lem}{Lemma}

%\theoremstyle{remark}
%\newtheorem{defn}{Definition}
%\newtheorem{rem}{Remark}
%\newtheorem{exmp}{Example}
%\newtheorem{note}{Note}

%endlocaldefs

\begin{document}

\begin{frontmatter}

\title{This is an Interesting Title}
%\title{\thanksref{t1}}
%\thankstext[id=t1]{}

\author[]{\inits{}\fnms{}~\snm{}\thanksref{t1}\ead{}}
\thankstext[type=corresp,id=t1]{}
%\thankstext[id=]{}
\address[]{\institution{}, ..., \cny{}}

%\runtitle{}  %use if necessary to change auto-generated           
%\runauthor{} %use if necessary to change auto-generated            

%\dedicated{}

\begin{abstract} % avoid notations; avoid citations; no exceeding 250 words
Abstract
\end{abstract}

\begin{keywords} % Alphabetical; not to repeat anything in the title
\kwd{}
\kwd{}
\kwd{}
\end{keywords}

\end{frontmatter}


\section{Introduction}\label{sec:intro}

The introduction needs to answer three questions:
1) Why should we care about this research (its importance)?
2) What have been done in the literature (identify a gap)?
3) What are new (your contributions)?


When citing literatures, be sure to understand the difference of
textual citations and parenthetical citations.
% Example of textual citation
In their influential work, \citet{carlin1992monte} introduced a new
approach to Bayesian modeling, which has been widely adopted in
statistical analysis.
% Example of parenthetical citation
Recent studies have shown the effectiveness of this method in various
applications \citep{gamado2014modelling,  gamado2017estimation}.
% Mixed usage example
According to \citet{carlin1992monte}, machine learning techniques
can be improved by incorporating domain-specific knowledge
\citep[see also][]{kotz2001laplace}.


A quick checklist on styles:
\begin{enumerate}
\item Each equation in display is part of a sentence, which needs to
  be punctuated as appropriate.
\item Put negative signs in math mode; otherwise it is just a dash.
\item Use the right citation command to distinguish textual and
  parenthetical citations.
\item Generate statistical graphics as vector graphics instead of
  raster graphics; the latter blurs when zoomed in.  
\item Put journal/book titles in title style in bibtex.
\item Protect upper cases in special words such as \{MCMC\},
  \{B\}ayesian, \{G\}aussian, etc.
\item Look up and cite the latest journal versions of preprints.
\end{enumerate}


\section{}\label{} % refer to section numbers by their labels in the paper

%% Supplementary Material is required for reproducibility check
%% List its contents and links to public repositories
\section*{Supplementary Material}
%% An example
Some additional simulation results and a compressed folder with the
code to simulate the settings with 5 moderators (5M), implement our
proposed methods, and some existing methods are provided as the online
Supplementary Material.

%% Appendices %%
%%%%%%%%%%%%%%%%%%%%%%
%\begin{appendix}
%\section{Appendix section}
%\end{appendix}

%% Acknowledgements, if needed
\section*{Acknowledgements}
This work was partially supported by the National Science Foundation
Division of Mathematical Sciences (Award number here).



\bibliographystyle{jds}  % using the JDS bib style
\bibliography{biblio}  % biblio.bib should store all your bibtex entries

\end{document} 